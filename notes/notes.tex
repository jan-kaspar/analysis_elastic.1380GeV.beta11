\input article
\input macros

\NumberSectionstrue

\def\baseDir{/afs/cern.ch/work/j/jkaspar/analyses/elastic/1380GeV,beta11}
\def\release{}


%----------------------------------------------------------------------------------------------------

\centerline{\SetFontSizesXX Elastic analysis, $\sqrt s = 2.76\un{TeV}$, $\be^* = 11\un{m}$ }
\vskip2mm
\centerline{version: {\it \number\day. \number\month. \number\year}}

%----------------------------------------------------------------------------------------------------
\section{Datasets}

\> DS1: 11 Feb 2013, fill 3549, RPs at $5\un{\si}$
\>> runs: 9009 and 9010
\>> run 9008 not used - \TODO{why ?}
\>> bunch 0

\> DS2: 12 Feb 2013, fill 3556, RPs at $22\un{\si}$
\>> runs: 9047 to 9050
\>> no signs of elastic events -- analysis stopped

\> DS3: 14 Feb 2013, fill 3564, RPs at $13\un{\si}$
\>> runs: 9078
\>> trigger on all available bunches

%----------------------------------------------------------------------------------------------------
\section{Data-taking conditions}

\> RunLog entries
\>> run 9008: \link{https://webafs01.cern.ch/totem-runlog/fcgi/runs/51743/}
\>> run 9009: \link{https://webafs01.cern.ch/totem-runlog/fcgi/runs/51726/}
\>> run 9010: \link{https://webafs01.cern.ch/totem-runlog/fcgi/runs/51727/}
\>> run 9078: \link{https://webafs01.cern.ch/totem-runlog/fcgi/runs/51750/}

\> OP LogBook
\>> DS1: http://elogbook.cern.ch/eLogbook/eLogbook.jsp?shiftId=1051885

\> emittance measurements
\>> DS1: \summary{emittances.ods}
\>> DS3: not available

$$
\hbox{beam divergence} = \sqrt{\ep_{\rm N} \over \be^* \ga}\ ,\qquad
\hbox{beam size} = \sqrt{\ep_{\rm N} \be^*\over \ga}\ ,\qquad
\hbox{vertex size} = {\hbox{beam size}\over \sqrt 2}
$$

\> rates as function of time: \plot{time_dependences/rates_vs_time.pdf}

%----------------------------------------------------------------------------------------------------
\section{Ntuples}

\> only events with even numbers used, events with odd numbers blinded

\subsection{Currently used}

\> /castor/cern.ch/totem/offline/Analysis/2013/Physics/9009\_9010/v\_4.0/,\\ /castor/cern.ch/totem/offline/Analysis/2013/Physics/9078/v\_4.0/
\>> RP/T1/T2 reconstruction
\>> using preliminary alignment corrections - only horizontal shifts and tilts (the large effect most likely due to optics treated as misalignment in order to correct for it!)

\subsection{Other}

\> /castor/cern.ch/user/j/jkaspar/reco/sr+hsx/
\>> only track-based alignment (sr+hsx)

%----------------------------------------------------------------------------------------------------
\section{Hit distributions}

\> \plot{hit_distributions/hit_distributions.pdf} : hit distributions before and after elastic selection

%----------------------------------------------------------------------------------------------------
\section{Alignment}

\> track-based alignment (relative among RPs):
\>> horizontal RP missing: track-based alignment not performed in the usual manner
\>> using track-based alignment results from previous runs, see my analysis-meeting presentation on 27 Feb 2013
\>>> expected uncertainty about 20 um

\> alignment w.r.t beam (using elastics, standard method)
\>> horizontal: uncertainty about 20 um
\>> vertical: uncertainty 50 um
\>> tilts: uncertainty 2 mrad

\> \plot{alignment/alignment.pdf} : \TODO{describe}
\> \plot{alignment/alignment_method_x.pdf} : \TODO{describe}
\> \plot{alignment/alignment_method_y.pdf} : \TODO{describe}

%----------------------------------------------------------------------------------------------------
\section{Optics}

\> studies by Frici
\>> use data /castor/cern.ch/user/j/jkaspar/reco/sr+hsx/

\> \plot{optics/optics_test_basic.pdf} : \TODO{describe}
\> \plot{optics/optics_test_summary.pdf} : \TODO{describe}
\> \plot{optics/theta_vs_phi.pdf} : \TODO{describe}


%----------------------------------------------------------------------------------------------------
\section{Reconstruction formulae}

\> Monte-Carlo study: \plot{reco_formulae/plot_formulae_graphs.pdf}
\>> \TODO{tune beam divergence and vertex size}: at the moment, angular resolution is good, but left-right vertex difference is too large in data wrt.~MC

\> chosen formulae
\>> $\th_x^*$: single arm = regr, double arm = regr (4 RP)
\>> $\th_y^*$: single arm = hit, double arm = LRavg-hit
\>> $x^*$: single arm = regr, double arm = LRavg-regr
\>> $y^*$: single arm = regr, double arm = LRavg-regr 

%----------------------------------------------------------------------------------------------------
\section{Resolution}

\> \TODO{resolution figures extracted from data -- to complement simulation results from the previous section}

%----------------------------------------------------------------------------------------------------
\section{Cuts/elastic tagging}

\> applied cuts 
\>> \plot{cuts/cut_1.pdf} : left-right collinearity in $\th_x^*$
\>> \plot{cuts/cut_2.pdf} : left-right collinearity in $\th_y^*$
\>> \plot{cuts/cut_5.pdf} : local position-angle correlation in vertical plane and right arm
\>> \plot{cuts/cut_6.pdf} : local position-angle correlation in vertical plane and left arm
\>> \plot{cuts/cut_7.pdf} : comparison of $x^*$ reconstructed from left and right arm (correlated to $\th_x^*$ due to optics imperfections)

\> other cuts, not applied
\>> \plot{cuts/cut_3.pdf} : right arm $x^*$ compatibility with expected size (modulo correlation with $\th_x^{*R}$)
\>> \plot{cuts/cut_4.pdf} : left arm $x^*$ compatibility with expected size (modulo correlation with $\th_x^{*L}$)
\>> \plot{cuts/cut_8.pdf} : comparison of $y^*$ reconstructed from left and right arm (correlated to $\th_y^*$ to allow for optics imperfections)

\> effect of different cut combinations (rows) on various correlation plots (columns)
\>> \plot{cuts/cut_matrix_DS1_45b_56t.pdf}: for DS1 and diagonal 45b--56t
\>> \plot{cuts/cut_matrix_DS1_45t_56b.pdf}: for DS1 and diagonal 45t--56b

%----------------------------------------------------------------------------------------------------
\section{Background}


%----------------------------------------------------------------------------------------------------
\section{Binning}


%----------------------------------------------------------------------------------------------------
\section{Acceptance correction}

\> \TODO{vertex smearing plays a role ?!}

\> \plot{acceptance_correction/acc_corr_phi_lab.pdf} : \TODO{describe}


%----------------------------------------------------------------------------------------------------
\section{Efficiency studies, pile-up, ...}

\subsection{Trigger, DAQ}

\subsection{3-out-of-4 efficiencies}

\> Why 70-80\% efficiency in the right arm?
\>> The same for both diagonals of DS1
\>> Effect different in DS3: very good efficiency in right near, possibly reduced eff. in right far
\>> 2D plots: inefficiency distributed all over the hit pattern

\> \plot{efficiencies/eff3outof4.pdf} : single-RP (in)efficiencies as function of $\th_y^*$
\> \plot{efficiencies/eff3outof4_2D.pdf} : single-RP (in)efficiencies as function of $\th_x^*$ and $\th_y^*$

\subsection{2-out-of-4 efficiencies}

\subsection{Pile-up}

\> \plot{efficiencies/pileup.pdf} : \TODO{describe}
\>> DS1: why so much higher in right arm (~40\%) than in left arm (~5\%)


\subsection{Showers}

\> \plot{mult_single_RP_2D.asy} : \TODO{describ}
\> \plot{mult_single_RP.asy} : \TODO{describ}
\> \plot{shower_sign_conf.asy} : \TODO{describ}

%----------------------------------------------------------------------------------------------------
\section{Unfolding of resolution effects}


%----------------------------------------------------------------------------------------------------
\section{Systematic uncertainties}


%----------------------------------------------------------------------------------------------------
\section{$t$-distributions}

\> \plot{t_distributions/t_dist.pdf} : \TODO{describe}

\bye
